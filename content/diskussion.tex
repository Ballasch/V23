\section{Diskussion}
\label{sec:diskussion}

\subsection{Wasserstoffatom}

Die diskreten Energieniveaus des Wasserstoffatoms werden gut durch den Kugelresonator simuliert, da das Frequenzspektrum ausgeprägte und scharfe Peaks besitzt. 
Die Messungen der Winkelverteilungen für alle betrachteten Peaks sind in diesem Abschnitt auch gut mit den vorhergesagten theoretischen Orbitalen vereinbar. Das $3D$ Orbital des Wasserstoffatoms konnte gut aufgelöst werden.
Aus diesem Grund ist der Kugelresonator ein gutes Modell für die den winkelabhängigen Teil der Wellenfunktion eines Wasserstoffatoms. Die Radialkomponente der Gesamtwellenfunktion kann ein Kugelresonator nicht modellieren.  
Des Weiteren deckt das Aufspalten der Peaks im Spektrum durch das Einführen einer Blende innerhalb des Resonators die theoretischen Erwartungen. Ein Zwischenring bricht die Kugelsymmetrie und löst damit die Entartung in $m$ auf. 
\subsection{Winkelverteilung des Wasserstoffmoleküls}

Bei der gemessenen Winkelverteilung der gekoppelten Kugelresonatoren konnte nur der dritte Peak eindeutig und der erste Peak nur näherungsweise zu einem Molekülorbital zugeordnet werden (siehe Abbildung \ref{fig:h2_2}). 
Dies kann zum Teil damit begründet werden, dass der zweite Peak sich in in einem Mischzustand befindet und damit eine Summe aus verschiedenen Orbitalen darstellt. Es haben sich also der erste Peak und der zweite Peak sich überlagert. \\
Des Weiteren konnte der Winkel bei diesem Versuchsaufbau nicht sehr präzise gemessen werden.\\
Der Zustand des dritten Peaks konnte jedoch eindeutig bestimmt werden. Dies kann damit begründet werden, dass die übrigen Peaks weit genug weg sind, um sich zu überlagern.
\subsection{Festkörpermodellierung}

Die Messergebnisse zu den Zylinderketten, welche einen eindimensionalen Festkörper simulieren sollen, folgen im Allgemeinen die theoretischen Erwartungen. Die Bandstruktur bei der Resonatorkette ohne ein Defekt weist genau den erwarteten Verlauf auf. 
Die Anzahl der scharfen Peaks in jedem Band entspricht der Anzahl der Zylinder in der Kette. Das kann gut mit den Elektronenzuständen in einem Festkörper verglichen werden. Hierbei ist der Spin der Elektronen zu vernachlässigen und von einem Elektron pro Zylinder auszugehen. \\
Die Messungen zu der modifizierten Resonatorkette entsprechen ebenfalls den Erwartungen. Durch alternierende Blenden in der Kette kommt es zu Veränderungen innerhalb eines Bandes. Es entstehen hierdurch beispielsweise Bandlücken. 
Durch alternierende Zylinderlängen haben sich die Bänder verschoben. Es kam also zu Veränderungen der gesamten Bandstruktur. Diese Veränderungen entsprechen den tatsächlichen Beobachtungen bei Festkörpern. 