\section{Zielsetzung}
\label{sec:ziel}

Ziel des Versuchs ist es quantenmechanische Strukturen wie das Wasserstoffatom, Wasserstoffmolekül und die Bandstruktur in eindimensionalen Festkörpern mit Hilfe von Analogien in der Akustik zu untersuchen und die Gemeinsamkeiten und Grenzen dieser Analogien zu untersuchen. Dazu werden akustische Experimente mit Hohlraumresonatoren und Zylindern aus Aluminium durchgeführt.

\section{Theorie}
\label{sec:theorie}

Für die quantenmechanischen Modellen können Analogien mit Hilfe der Akustik geschaffen werden, im Folgenden werden die quantenmechanischen Grundlagen für die einzelnen Modelle erläutert und die Gemeinsamkeiten und Unterschiede zu den akustischen Experimenten benannt und begründet. 

\subsection{Das Wasserstoffatom}
\label{sec:H}

Das Wasserstoffatom ist das simpelste Atom. Es besteht aus einem Proton im Kern und einem Elektron in der Hülle. Die zeitunabhängige Schrödingergleichung für dieses System lautet:

\begin{equation}
    \hat{H} \, \Psi \! \left( \vec{r} \right) = - \frac{\hbar^2}{2 m} \! \Laplace \, \Psi \! \left( \vec{r} \right)- \frac{e^2}{4 \pi \epsilon_0 r} \, \Psi \! \left( \vec{r} \right) = E \, \Psi \! \left( \vec{r} \right)
    \label{eqn:schroedinger}
\end{equation}

Dabei ist $\Psi \! \left( \vec{r} \right)$ ist die Wellenfunktion, $E$ die Gesamtenergie und $\hat{H}$ der Hamiltonoperator. Für ein Elektron im Wasserstoffatom lautet $\hat{H}$:

\begin{equation}
    \hat{H} = - \frac{\hat{p}^2}{2 m} - \frac{e^2}{4 \pi \epsilon_0 r}
    \label{eqn:H_H}
\end{equation}

Hierbei ist $\hat{p}$ der Impulsopertaor, $\hbar$ das gekürzte planksche Wirkungsquantum, $m$ die Masse des Elektrons, $e$ die Elementarladung und $\epsilon_0$ die elektrische Feldkonstante. Aufgrund der Kugelsymmteire des Systemes werden Kugelkoordinaten verwendet, dort lautet der Laplace-Operator $\Laplace$ für eine Beispielfunktion $f$:

\begin{equation}
    \Laplace f = \frac{1}{r^2} \frac{\partial}{\partial r} \left( r^2 \frac{\partial f}{\partial r} \right) + \frac{1}{r^2 \sin \theta} \frac{\partial}{\partial \theta} \left( \sin \theta \frac{\partial f}{\partial \theta} \right) + \frac{1}{r^2 \sin^2 \theta} \frac{\partial^2 f}{\partial \phi^2}
    \label{eqn:laplace}
\end{equation}

Um die Schrödingergleichung zu lösen wird die Wellenfunktion $\Psi$ mit dem Seperationsansatz in einen Radialteil $R_{nl}$ und einen Winkelanteil $\Phi_{lm}$ aufgeteilt:

\begin{equation}
    \Psi_{nlm} (\vec{r}) = R_{nl}(r) \, \Phi_{lm}(\theta, \phi)
    \label{eqn:seperation}
\end{equation}

Für das Wasserstoffatom gibt es 3 Quantenzahlen namens $n, l, m$. Dabei ist $n$ die Hauptquantenzahl, $l$ die Nebenquantenzahl und $m$ die Magnetquantenzahl. Für die Quantenzahlen gilt

\begin{align*}
    n &\in \mathbb{N} \\
    l &\in \mathbb{N}_0 \\
    m &\in \mathbb{Z}
\end{align*}

und

\begin{align*}
    l &< n \\
    |m| &\leq l .
\end{align*}

Mit dem Seperationsansatz entstehen zwei entkoppelte Differentialgleichungen. Für dieses Experiment ist jedoch nur die Lösung des Winkelanteils interessant, da nur dieser in dem akustischen Modell modelliert werden kann. Dies führt für den Winkelanteil zu folgender Differentialgleichung:

\begin{equation}
    \Laplace_{\theta, \phi} \, \Phi_{lm} (\theta, \psi) = - l (l+1) \Phi_{lm}
    \label{eqn:eigenwert}
\end{equation}

Dabei ist $\Laplace_{\theta, \phi}$ der Winkelanteil des Laplaceoperators $\Laplace$ in Kugelkoordinaten. Der Term mit $l (l+1)$ kommt in der Herleitung dadurch zustande, dass $- \hbar^2 \Laplace_{\theta, \phi} = \hat{L}^2$ entspricht, wobei $\hat{L}$ der Drehimpulsoperator ist. Für $\hat{L}^2$ gilt folgende Eigenwertgleichung:

\begin{equation}
    \hat{L}^2 \ket{\psi} = \hbar^2 l(l+1) \ket{\psi}
    \label{eqn:L}
\end{equation}

Die Eigenwertgleichung \eqref{eqn:eigenwert} kann mit Hilfe der \textit{Kugelflächenfunktionen} $Y_{lm} (\theta, \phi)$ gelöst werden und diese ergeben sich zu:

\begin{equation}
    Y_{lm} (\theta, \phi) = \frac{1}{\sqrt{2 \pi}} N_{lm} P_{lm} (\cos \theta) e^{im\phi}
    \label{eqn:kugelflaechen}
\end{equation}

Hierbei bezeichnet $P_{lm}$ die zugeordneten \textit{Legendrepolynome}

\begin{equation}
    P_{lm} (x) = \frac{(-1)^m}{2^l \, l!} \left( 1-x^2 \right)^{\! \frac{m}{2}} \frac{d^{\, l+m}}{dx^{l+m}} \left( x^2 - 1 \right)^l
    \label{eqn:legendre}
\end{equation}

und $N_{lm}$ den Normierungsfaktor

\begin{equation}
    N_{lm} = \sqrt{\frac{2 \, l +1}{2} \cdot \frac{(l-m)!}{(l+m)!}} .
    \label{eqn:normierung}
\end{equation}

Die Eigenenergiewerte des Wasserstoffatoms betragen:

\begin{equation}
    E_n = - \frac{e^2}{8 \pi \epsilon_0 a_0} \cdot \frac{1}{n^2}
    \label{eqn:H_energien}
\end{equation}

Dabei bezeichnet $a_0 = \frac{4 \pi \epsilon_0 \hbar^2}{m e^2}$ den Bohrschen Radius. In Gleichung \eqref{eqn:H_energien} ist auffällig, dass die Eigenenergiewerte beim Wasserstoffatom eine Entartung in $m$ aufzeigen, die durch die sphärische Symmetrie resultiert. Diese Entartung kann durch ein anlegen eines äußeren Magnetfeldes aufgehoben werden. Durch die Felder wird die Symmetrie gebrochen. Dieser Effekt wird \textit{Zeemanneffekt} genannt. Die Entartung in $l$ ist jedoch ein Resultat aus dem $\frac{1}{r}$ - Potential.

\subsection{Das Wasserstoffmolekül}
\label{sec:H2}



\subsection{Der 1-dim Festkörper}
\label{sec:festkoerper}



\subsection{Grundlagen der Akustik}
\label{sec:akustik}



\subsection{Analogie zum Wasserstoffatom und -molekül}
\label{sec:analogien}

Im Folgenden werden die Analogien zwischen akustischen Experimenten und dem Modell des Wasserstoffatom und Wasserstoffmolekül dargestellt.

\subsubsection{Wasserstoffatom}
\label{sec:ana-H}



\subsubsection{Wasserstoffmolekül}
\label{sec:ana-H2}



\subsection{Analogie zum 1-dim Festkörpers}
\label{ana-fest}

