\section{Zielsetzung}
\label{sec:ziel}

Ziel des Versuchs ist es quantenmechanische Strukturen wie das Wasserstoffatom, Wasserstoffmolekül und die Bandstruktur in eindimensionalen Festkörpern mit Hilfe von Analogien in der Akustik zu untersuchen und die Gemeinsamkeiten und Grenzen dieser Analogien zu untersuchen. Dazu werden akustische Experimente mit Hohlraumresonatoren und Zylindern aus Aluminium durchgeführt.

\section{Theorie}
\label{sec:theorie}

Für die quantenmechanischen Modellen können Analogien mit Hilfe der Akustik geschaffen werden, im Folgenden werden die quantenmechanischen Grundlagen für die einzelnen Modelle erläutert und die Gemeinsamkeiten und Unterschiede zu den akustischen Experimenten benannt und begründet. 

\subsection{Das Wasserstoffatom}
\label{sec:H}

Das Wasserstoffatom ist das simpelste Atom. Es besteht aus einem Proton im Kern und einem Elektron in der Hülle. Die zeitunabhängige Schrödingergleichung für dieses System lautet:

\begin{equation}
    \hat{H} \, \Psi \! \left( \vec{r} \right) = - \frac{\hbar^2}{2 m} \! \Laplace \, \Psi \! \left( \vec{r} \right)- \frac{e^2}{4 \pi \epsilon_0 r} \, \Psi \! \left( \vec{r} \right) = E \, \Psi \! \left( \vec{r} \right)
    \label{eqn:schroedinger}
\end{equation}

Dabei ist $\Psi \! \left( \vec{r} \right)$ ist die Wellenfunktion, $E$ die Gesamtenergie und $\hat{H}$ der Hamiltonoperator. Für ein Elektron im Wasserstoffatom lautet $\hat{H}$:

\begin{equation}
    \hat{H} = - \frac{\hat{p}^2}{2 m} - \frac{e^2}{4 \pi \epsilon_0 r}
    \label{eqn:H_H}
\end{equation}

Hierbei ist $\hat{p}$ der Impulsopertaor, $\hbar$ das gekürzte planksche Wirkungsquantum, $m$ die Masse des Elektrons, $e$ die Elementarladung und $\epsilon_0$ die elektrische Feldkonstante. Aufgrund der Kugelsymmteire des Systemes werden Kugelkoordinaten verwendet, dort lautet der Laplace-Operator $\Laplace$ für eine Beispielfunktion $f$:

\begin{equation}
    \Laplace f = \frac{1}{r^2} \frac{\partial}{\partial r} \left( r^2 \frac{\partial f}{\partial r} \right) + \frac{1}{r^2 \sin \theta} \frac{\partial}{\partial \theta} \left( \sin \theta \frac{\partial f}{\partial \theta} \right) + \frac{1}{r^2 \sin^2 \theta} \frac{\partial^2 f}{\partial \phi^2}
    \label{eqn:laplace}
\end{equation}

Um die Schrödingergleichung zu lösen wird die Wellenfunktion $\Psi$ mit dem Seperationsansatz in einen Radialteil $R_{nl}$ und einen Winkelanteil $\Phi_{lm}$ aufgeteilt:

\begin{equation}
    \Psi_{nlm} (\vec{r}) = R_{nl}(r) \, \Phi_{lm}(\theta, \phi)
    \label{eqn:seperation}
\end{equation}

Für das Wasserstoffatom gibt es 3 Quantenzahlen namens $n, l, m$. Dabei ist $n$ die Hauptquantenzahl, $l$ die Nebenquantenzahl und $m$ die Magnetquantenzahl. Für die Quantenzahlen gilt

\begin{align*}
    n &\in \mathbb{N} \\
    l &\in \mathbb{N}_0 \\
    m &\in \mathbb{Z}
\end{align*}

und

\begin{align*}
    l &< n \\
    |m| &\leq l .
\end{align*}

Mit dem Seperationsansatz entstehen zwei entkoppelte Differentialgleichungen. Für dieses Experiment ist jedoch nur die Lösung des Winkelanteils interessant, da nur dieser in dem akustischen Modell modelliert werden kann. Dies führt für den Winkelanteil zu folgender Differentialgleichung:

\begin{equation}
    \Laplace_{\theta, \phi} \, \Phi_{lm} (\theta, \psi) = - l (l+1) \Phi_{lm}
    \label{eqn:eigenwert}
\end{equation}

Dabei ist $\Laplace_{\theta, \phi}$ der Winkelanteil des Laplaceoperators $\Laplace$ in Kugelkoordinaten. Der Term mit $l (l+1)$ kommt in der Herleitung dadurch zustande, dass $- \hbar^2 \Laplace_{\theta, \phi} = \hat{L}^2$ entspricht, wobei $\hat{L}$ der Drehimpulsoperator ist. Für $\hat{L}^2$ gilt folgende Eigenwertgleichung:

\begin{equation}
    \hat{L}^2 \ket{\psi} = \hbar^2 l(l+1) \ket{\psi}
    \label{eqn:L}
\end{equation}

Die Eigenwertgleichung \eqref{eqn:eigenwert} kann mit Hilfe der \textit{Kugelflächenfunktionen} $Y_{lm} (\theta, \phi)$ gelöst werden und diese ergeben sich zu:

\begin{equation}
    Y_{lm} (\theta, \phi) = \frac{1}{\sqrt{2 \pi}} N_{lm} P_{lm} (\cos \theta) e^{im\phi}
    \label{eqn:kugelflaechen}
\end{equation}

Hierbei bezeichnet $P_{lm}$ die zugeordneten \textit{Legendrepolynome}

\begin{equation}
    P_{lm} (x) = \frac{(-1)^m}{2^l \, l!} \left( 1-x^2 \right)^{\! \frac{m}{2}} \frac{d^{\, l+m}}{dx^{l+m}} \left( x^2 - 1 \right)^l
    \label{eqn:legendre}
\end{equation}

und $N_{lm}$ den Normierungsfaktor

\begin{equation}
    N_{lm} = \sqrt{\frac{2 \, l +1}{2} \cdot \frac{(l-m)!}{(l+m)!}} .
    \label{eqn:normierung}
\end{equation}

Die Eigenenergiewerte des Wasserstoffatoms betragen:

\begin{equation}
    E_n = - \frac{e^2}{8 \pi \epsilon_0 a_0} \cdot \frac{1}{n^2}
    \label{eqn:H_energien}
\end{equation}

Dabei bezeichnet $a_0 = \frac{4 \pi \epsilon_0 \hbar^2}{m e^2}$ den Bohrschen Radius. In Gleichung \eqref{eqn:H_energien} ist auffällig, dass die Eigenenergiewerte beim Wasserstoffatom eine Entartung in $m$ aufzeigen, die durch die sphärische Symmetrie resultiert. Diese Entartung kann durch ein anlegen eines äußeren Magnetfeldes aufgehoben werden. Durch die Felder wird die Symmetrie gebrochen. Dieser Effekt wird \textit{Zeemanneffekt} genannt. Die Entartung in $l$ ist jedoch ein Resultat aus dem $\frac{1}{r}$ - Potential. Zusätzlich führt die Berücksichtigung von Spin und relativistischen Beiträgen zu weiteren Aufspaltungen, diese wurden hierbei jedoch nicht berücksichtigt.

\subsection{Das Wasserstoffmolekül}
\label{sec:H2}

Das Wasserstoffmolekül $H_2$ besteht aus 2 Wasserstoffatomen und ist damit das einfachste neutrale Molekül. Es besteht also aus 2 positiven Protonen und 2 Elektronen. Dieses Problem ist jedoch nicht analytisch lösbar. Jedoch existieren Näherungen, um die Wellenfunktion des Systems zu approximieren. Die zu lösende Schrödingergleichung lautet:

\begin{equation}
    E \, \Psi (1, 2) = \left( \hat{H}_1 + \hat{H}_2 - \frac{e^2}{r_{a2}} - \frac{e^2}{r_{b1}} - \frac{e^2}{r_{12}} - \frac{e^2}{R_{ab}} \right) \, \Psi (1, 2)
    \label{eqn:H2}
\end{equation}

Hierbei sind $\hat{H}_{1,2}$ die Hamiltonoperatoren der einzelnen Wasserstoffatome, und die anderen Variablen beschreiben die Abstände der Elektronen 1 und 2 zu den Kernen a und b aus Abbildung \ref{fig:H2}.

%Bild muss ich noch einfügen bzw selber zeichnen

Da die Elektronen Fermionen sind unterliegen sie dem \textit{Pauli-Verbot} und dies muss auch die Wellenfunktion erfüllen. Also muss die Gesamtwellenfunktion antisymmetrisch sein. Die Gesamtwellenfunktion $\Psi (1,2)$ besteht aus einer Ortswellenfunktion $\tilde{\Psi} (1,2)$ und einer Spinwellenfunktion $S (1,2)$:

\begin{equation}
    \Psi (1,2) = \tilde{\Psi} (1,2) \cdot S (1,2)
    \label{eqn:gesamt}
\end{equation}

Dadurch muss die Spinwellenfunktion antisymmetrisch sein, falls die Ortswellenfunktion symmetrisch ist und vice versa. Dadurch ergeben sich 4 mögliche Wellenfunktionen:

\begin{align}
    \Psi_{t1} (r_1, r_2) &= \, \uparrow_1 \uparrow_2 \left( \Psi_a (r_1) \Psi_b (r_2) - \Psi_a (r_2) \Psi_b (r_1) \right) \\
    \Psi_{t2} (r_1, r_2) &= \, \downarrow_1 \downarrow_2 \left( \Psi_a (r_1) \Psi_b (r_2) - \Psi_a (r_2) \Psi_b (r_1) \right) \\
    \Psi_{t3} (r_1, r_2) &= \frac{1}{\sqrt{2}} \left( \uparrow_1 \downarrow_2 + \uparrow_2 \downarrow_1 \right) \left( \Psi_a (r_1) \Psi_b (r_2) - \Psi_a (r_2) \Psi_b (r_1) \right) \\
    \Psi_{s} (r_1, r_2) &= \frac{1}{\sqrt{2}} \left( \uparrow_1 \downarrow_2 - \uparrow_2 \downarrow_1 \right) \left( \Psi_a (r_1) \Psi_b (r_2) + \Psi_a (r_2) \Psi_b (r_1) \right)
\end{align}

Dabei bilden die $\Psi_t$ Wellenfunktionen das \textit{Triplett} mit antisymmetrischer Ortswellenfunktion und symmetrischer Spinwellenfunktion, dies wird als \textit{Orthowasserstoff} bezeichnet. $\Psi_s$ ist das \textit{Singulett} mit antisymmetrischer Spinwellenfunktion und symmetrischer Ortswellenfunktion und wird als \textit{Parawasserstoff} bezeichnet. 

\subsection{Der 1-dim Festkörper}
\label{sec:festkoerper}

Beim Festkörper kommt es durch das Pauli-Prinzip, das beschreibt dass 2 Fermionen wie Elektronen sich nicht gleichzeitig im selben Zustand befinden dürfen, im Gegensatz zum Wasserstoffatom nicht zu scharfen Spektrallinien sondern zu Energiebändern, die aus den \textit{erlaubten Zonen} bestehen. Zwischen den Energiebändern gibt es Bandlücken, die aus den \textit{verbotenen Zonen} bestehen. Um einen Festkörper zu modellieren werden Kastenpotentiale mit periodischen Randbedingungen verwendet. Dadurch ergibt sich in der Dispersionsrelation $E(\vec{k})$ in erster Näherung eine Proportionalität zu $k^2$. Die Dispersionsrelation beträgt:

\begin{equation}
    E(k) = \frac{\hbar^2 k^2}{2 m}
    \label{eqn:disp}
\end{equation}

In einem eindimensionalen Festkörper wird ein Elektron in einem periodischen Potential $U(x)$ betrachtet. Das Potential und die Wellenfunktion in eine Fourierreihe entwickelt ergibt dann:

\begin{align}
    U (x) &= \sum_G U_G \cdot e^{iGx} \\
    \Psi (x) &= \sum_k C_k \cdot e^{ikx}
\end{align}

Aus den periodischen Randbedingungen ergibt sich dann $k = \frac{2 \pi}{L} n$. Dies eingesetzt in die Schrödingergleichung ergibt dann:

\begin{equation}
    E \, \Psi (x) = \left( - \frac{\hbar^2}{2m} \Laplace + U (x) \right) \Psi (x)
    \label{eqn:fest_schroed}
\end{equation}

Für ein Elektron ergibt sich dann folgende Gleichung:

\begin{equation}
    \left( \frac{\hbar^2 k^2}{2m} - E \right) C_k + \sum_G U_G C_{k-G} = 0
    \label{eqn:fest_el}
\end{equation}

Die Darstellung der Dispersionsrelation muss jedoch auf ein reduziertes Zonenschema mit Wellenvektor beschränkt auf $- \frac{\pi}{a} \leq k \leq \frac{\pi}{a}$ eingegrenzt werden. Das vollständige Zonenschemaergibt sich durch periodisches aneinander Reihen dieser Dispersionsrelation vom reduzierten Zonenschema. Die Zone, die das reduzierte Zonenschema umfasst, wird auch als die \textit{erste Brillouin-Zone} bezeichnet.  

Jedoch sind Festkörper in der Realität nicht zu $100 \%$ reine Kristalle und besitzen oft Defekte. Diese Defekte werden häufig durch Fehlstellen oder Fremdatome in der Kristallstruktur ausgelöst und müssen durch veränderte Potentialtöpfe modelliert werden

\subsection{Grundlagen der Akustik}
\label{sec:akustik}



\subsection{Analogie zum Wasserstoffatom und -molekül}
\label{sec:analogien}

Im Folgenden werden die Analogien zwischen akustischen Experimenten und dem Modell des Wasserstoffatom und Wasserstoffmolekül dargestellt.

\subsubsection{Wasserstoffatom}
\label{sec:ana-H}



\subsubsection{Wasserstoffmolekül}
\label{sec:ana-H2}



\subsection{Analogie zum 1-dim Festkörpers}
\label{ana-fest}

